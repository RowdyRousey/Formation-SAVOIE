\begin{frame}{Repérer les différents équipements dans la baie SAVOIE}
\begin{block}{En salle technique}
    \begin{itemize}
        \item Segment OPS et PREOPS
        \item Les ESX
        \item Les SAN
    \end{itemize}
\end{block}
\end{frame}

\begin{frame}{Interfaces Web à disposition}
    \begin{itemize}
        \item Interface relative à un ESX
        \item Interface relative au vCenter
        \item Interface relative au SAN 
    \end{itemize}
\end{frame}

\begin{frame}{Adressage réseaux de la plateforme}
A l'aide des interfaces Web, remplir les tableaux suivants avec les adresses IP virtuelles des ESX.
\\
\begin{center}
 \begin{tabular}{|c || c | c | c | c|} 
 \hline
  & GESTION & vMOTION & STOCKAGE & SYSTEME \\ [0.5ex] 
 \hline
 valthorens &  &  & & \\ [0.5ex]
 \hline
 courchevel &  &  & &  \\ [0.5ex]
 \hline
 meribel &  &  &  &  \\ [0.5ex]
 \hline
 saulire &  &  &  &  \\ [0.5ex]
 \hline
 lachambre &  &  &  &  \\ [0.5ex]
 \hline
 latania  &  &  &  &  \\ [0.5ex]
 \hline
\end{tabular}
\end{center}
\end{frame}

\begin{frame}{Exercice}
\begin{block}{Enoncé}
\begin{itemize}
\item Création d'une VM \textit{\textbf{magasinier}} utilisant le datastore distant \textbf{\textit{entrepot}}
\end{itemize}
\end{block}
\begin{block}{Caractéristique de la VM magasinier}
\begin{itemize}
\item 2 GB de RAM
\item Utilise un OS Linux
\item Est connecté au moins à un réseau \textbf{\textit{ligne2prod}}
\end{itemize}
\end{block}
\begin{block}{Caractéristique du datastore entrepot}
\begin{itemize}
\item 25 Go Thin provisioning
\item Systeme de fichiers VMFS
\end{itemize}
\end{block}
\end{frame}

\begin{frame}{Jouons un peu avec notre VM}
\begin{block}{Etat de santé de la VM}
\begin{itemize}
\item Graphiques montrant la charge CPU et l'utilisation de la RAM
\end{itemize}
\end{block}
\begin{block}{Test du vMotion}
\begin{itemize}
\item Migrer la VM magasinier sur un autre ESX. 
\item Regarder le temps de traitement, la charge réseau sur les ESX, les logs.
\end{itemize}
\end{block}
\end{frame}

\begin{frame}{VM et datastore}
\begin{block}{Une VM, c'est quoi déjà ?}
\begin{itemize}
\item Où peut-on trouver concrètement la VM magasinier sur le SAN ? 
\item Renommer le fichier vm-magasinier.VMDK en vm-marsouin.VMDK. Que se passe-t-il ? Pourquoi ? 
\item Renommer le fichier correctement
\end{itemize}
\end{block}
\end{frame}

\begin{frame}{Exercice}
\begin{block}{Enoncé}
\begin{itemize}
\item Dans la VM, créer sur le bureau un fichier texte nommé "Marseille" et y inscrire dedans une phrase quelconque.
\item Voyons comment faire un snapshot de la VM (expliquer ce qu'est un snapshot rapidement)
\item Regardons dans le datastore entrepôt le fichier créé
\item Simuler une erreur d'un agent en détruisant le fichier Marseille.
\item Utiliser la gestion des snapshots pour revenir à l'état initial de la VM et ainsi s'apercevoir que l'erreur a été corrigée.
\end{itemize}
\end{block}
\begin{alertblock}{Attention}
\begin{itemize}
\item L'utilisation des snapshots n'est pas une méthode de sauvegarde de VM
\end{itemize}
\end{alertblock}
\end{frame}

\begin{frame}{Cas de pannes : ESX}
\begin{block}{ESX}
\begin{itemize}
\item Eteindre proprement l'ESX sur lequel tourne la VM magasinier. Que se passe-t-il ?
\item Regardons les logs
\item Redémarrer proprement l'ESX et attendre la remise en conformité de la plate-forme
\end{itemize}
\end{block}

\begin{block}{Cas de pannes : ESX ft vMotion}
\begin{itemize}
\item Créer un cluster et y inclure la VM Magasinier
\item Eteindre proprement l'ESX sur lequel tourne la VM magasinier. Que se passe-t-il ?
\item Regardons les logs
\item Redémarrer proprement l'ESX et attendre la remise en conformité de la plate-forme
\end{itemize}
\end{block}
\end{frame}

\begin{frame}{Cas de pannes 2}
\begin{block}{Cas de pannes : SAN}
\begin{itemize}
\item Eteindre proprement le SAN maître. Que se passe-t-il dans la VM ?
\item Si on ne dispose que d'un SAN sur la plateforme, éteindre proprement le SAN. Que se passe-t-il ?
\item Regardons les logs
\item Redémarrer proprement le SAN et attendre la remise en conformité de la plate-forme
\end{itemize}
\end{block}
\end{frame}

\begin{frame}{Cas de pannes 3}
\begin{block}{Cas de pannes : Réseaux}
\begin{itemize}
\item Repérer une NIC relative au VLAN STOCKAGE. La déconnecter du switch.
\item Que se passe-t-il au niveau de la VM ?
\end{itemize}
\end{block}

\begin{block}{Cas de pannes : Réseaux}
\begin{itemize}
\item Réperer une NIC relative au VLAN vMotion. La déconnecter du switch.
\item Que se passe-t-il au niveau de la VM ? 
\item Eteindre l'ESX hébergeant la VM magasinier. Que devrait-il se passer ? Que se passe-t-il ? Pourquoi ?
\end{itemize}
\end{block}
\end{frame}

\begin{frame}{Avant de partir ...}
\begin{block}{Remise en conformité de la plateforme}
\begin{itemize}
\item Supprimer la VM magasinier du datastore
\item Supprimer le cluster
\item Supprimer les réseaux créés
\item Supprimer toutes les modifications que l'on a apportées sur les SAN
\item Vérifier que les ESX et les SAN sont allumés
\item Vérifier les branchements réseaux
\end{itemize}
\end{block}
\end{frame}