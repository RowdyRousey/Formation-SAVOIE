\begin{frame}{MESO}
\begin{block}{Une MESO progressive}
\begin{itemize}
\item 1er Octobre : MESO SAVOIE (sans applicatifs)
\item 1er Novembre : MESO WikiFF sur positions et CDS via ATM2/SAVOIE
\end{itemize}
\end{block}
\begin{block}{Une MO répartie sur deux superviseurs}
\begin{itemize}
\item Les superviseurs RVU (resp CAW) utilisent leur casquette ODA (resp CAU) pour superviser SAVOIE
\item Les superviseurs RVU (resp CAW) utilisent leur casquette ODR (resp WAN) pour superviser ATM2
\item Pour chaque applicatif tournant dans la VM, le superviseur reste inchangé. Ex : WikiFF (RVU), 4Me (CAW)
\item Pour le début, il n'y aura qu'un SPV de formé \textit{a minima}. 
\end{itemize}
\end{block}
\end{frame}

\begin{frame}{Que doit faire le superviseur ?}
\begin{block}{Processus en trois temps}
\begin{itemize}
\item Identifier
\item Communiquer
\item Agir
\end{itemize}\pause
\end{block}
\begin{block}{Outils à la disposition des superviseurs}
\begin{itemize}
\item La station d'administration LATANIA
\item Le supervision applicative (WIP)
\item SIAM (WIP)
\end{itemize}
\end{block}
\end{frame}

\begin{frame}{Identifier et Communiquer}
\begin{block}{Identifier l'incident}
\begin{itemize}
\item Identifier une panne de composants (ESX, SAN, disque, etc)
\item Identifier la "bonne santé" d'une VM (i.e savoir si le problème vient de SAVOIE ou de l'application)
\end{itemize}\pause
\end{block}
\begin{block}{Communiquer}
\begin{itemize}
\item Aux utilisateurs l'impact et le temps de rétablissement
\item à la MS les infos les plus pertinentes possibles pour aider à la résolution du problème
\end{itemize}
\end{block}
\end{frame}

\begin{frame}{Agir}
\begin{block}{Agir sur une VM}
\begin{itemize}
\item Arrêter/Démarrer une VM dégradée ou en panne
\end{itemize}
\end{block}
\begin{block}{Agir sur les ESX}
\begin{itemize}
\item \textcolor{red}{Aucun changement d'ESX n'est fait par la MO}
\end{itemize}
\end{block}
\begin{block}{Agir sur les SAN}
\begin{itemize}
\item \textcolor{red}{Aucun changement de disque SAN n'est fait par la MO}
\item \textcolor{red}{Aucun changement de SAN n'est fait par la MO}
\end{itemize}
\end{block}
\end{frame}